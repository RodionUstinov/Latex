\documentclass[a5paper,12pt]{book}[2018/02/23]
\usepackage[utf8]{inputenc}
\usepackage[english,russian]{babel}
\usepackage{indentfirst}
\usepackage{misccorr}
\usepackage{graphicx}
\usepackage{amsmath}
\begin{document}
Мозг- это наиболее сложный и наименее изученный человеческий орган. Мы многого о нем не знаем, но тем не менее вот несколько фактов о нем. 

1. Нервные импульсы двигаются со скоростью 270 км/ч.

2. Для работы мозгу требуется столько же энергии, сколько 10-ваттной лампочке.

3. Клетка человеческого мозга может хранить в пять раз больше информации, чем любая энциклопедия.

4. Мозг использует 20\% всего кислорода, который поступает в кровеносную систему.

5. Ночью мозг гораздо активнее, чем днем.

6. Ученые говорят, что чем выше уровень IQ, тем чаще люди видят сны.

7. Информация проходит по разным нейронам с разной скоростью.

8. Сам мозг не чувствует боли.

9. На 80\% мозг состоит из воды.

{\bf Правило Лопиталя:}
\[
\lim_{n \to \infty}
\frac{f(x)}{g(x)}=
\frac{f'(x)}{g'(x)}
\]
\end{document}


